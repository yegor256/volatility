% Copyright (c) 2012, Yegor Bugayenko
% All rights reserved.
%
% Redistribution and use in source and binary forms, with or without
% modification, are permitted provided that the following conditions
% are met: 1) Redistributions of source code must retain the above
% copyright notice, this list of conditions and the following
% disclaimer. 2) Redistributions in binary form must reproduce the above
% copyright notice, this list of conditions and the following
% disclaimer in the documentation and/or other materials provided
% with the distribution. 3) Neither the name of Yegor Bugayenko nor
% the names of other contributors may be used to endorse or promote
% products derived from this software without specific prior written
% permission.
%
% THIS SOFTWARE IS PROVIDED BY THE COPYRIGHT HOLDERS AND CONTRIBUTORS
% "AS IS" AND ANY EXPRESS OR IMPLIED WARRANTIES, INCLUDING, BUT
% NOT LIMITED TO, THE IMPLIED WARRANTIES OF MERCHANTABILITY AND
% FITNESS FOR A PARTICULAR PURPOSE ARE DISCLAIMED. IN NO EVENT SHALL
% THE COPYRIGHT HOLDER OR CONTRIBUTORS BE LIABLE FOR ANY DIRECT,
% INDIRECT, INCIDENTAL, SPECIAL, EXEMPLARY, OR CONSEQUENTIAL DAMAGES
% (INCLUDING, BUT NOT LIMITED TO, PROCUREMENT OF SUBSTITUTE GOODS OR
% SERVICES; LOSS OF USE, DATA, OR PROFITS; OR BUSINESS INTERRUPTION)
% HOWEVER CAUSED AND ON ANY THEORY OF LIABILITY, WHETHER IN CONTRACT,
% STRICT LIABILITY, OR TORT (INCLUDING NEGLIGENCE OR OTHERWISE)
% ARISING IN ANY WAY OUT OF THE USE OF THIS SOFTWARE, EVEN IF ADVISED
% OF THE POSSIBILITY OF SUCH DAMAGE.
%
\documentclass[12pt]{article}
    \usepackage{tikz}
        \usetikzlibrary{shapes,arrows,shadows,plotmarks}
        \usetikzlibrary{decorations.pathmorphing}
        \usetikzlibrary{positioning}
        \usetikzlibrary{fit}

\begin{document}
    \setlength{\parindent}{0pt}
    \setlength{\parskip}{1em}

\title{Volatility of Source Code}
\author{Yegor Bugayenko}
\maketitle

\section{Introduction}

    Volatility of source code is an experimental metric that
    shows how big is the difference between actively and rarely changed (dead)
    code. It is to prove that a big percentage of dead code is
    an indicator of knowledge transfer problems in a project.

\section{Details}

    First, it is observed how often every source code file $f_i \in F$ was changed
    during a lifetime of a project. Consider an example project with 16 files,
    where the most frequently changed file $f_{10}$ was changed nine times:

    \begin{eqnarray}
    F &=&  \{f_0, f_2, \dots, f_{10}\} \\
    c(f_{10}) &=& 9
    \end{eqnarray}

    \begin{figure}[ht]
    \centering
    \immediate\write18{php ./observed.php > fig-1.tex}
    \input{fig-1.tex}
    \end{figure}

    Then, files are reordered according to their popularity into a new set $F'$,
    where the most popular file $f_{10}$ takes leading position because $c(f_{10})$ is
    the maximum, etc.:

    \begin{figure}[ht]
    \centering
    \immediate\write18{php ./ordered.php > fig-2.tex}
    \input{fig-2.tex}
    \end{figure}

    Then, $j$ and $c(x)$ are transformed into $x$ and $p(x)$ as:

    \begin{eqnarray}
    x &=& j / \lceil j \rceil \\
    p(x) &=& c(f'_{x \gets j}) / \lceil c(f) \rceil
    \end{eqnarray}

    \begin{figure}[ht]
    \centering
    \immediate\write18{php ./normalized.php > fig-3.tex}
    \input{fig-3.tex}
    \end{figure}

    Then, mean $\mu$ is calculated as:

    \begin{eqnarray}
    \mu = \frac{\displaystyle\sum {x \times p(x)}}{\displaystyle\sum p(x)}
    \end{eqnarray}

    Finally, variance is calculated as:

    \begin{eqnarray}
    Var(x) = \frac{\displaystyle\sum {|x - \mu|^2 \times p(x)}}{\displaystyle\sum p(x)}
    \end{eqnarray}

    For the example above the numbers are:

    \begin{figure}[ht]
    \centering
    \immediate\write18{php ./numbers.php > fig-4.tex}
    \input{fig-4.tex}
    \end{figure}

    Variance $Var(x)$ is called ``source code volatility''.

\section{Alternatives}

    Similar calculations can be done:

    \begin{itemize}
        \item with lines of code instead of files;
        \item with unique file editors instead of file changes.
    \end{itemize}

\end{document}
