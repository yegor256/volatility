% Copyright (c) 2012-2020, Yegor Bugayenko
% All rights reserved.
%
% Redistribution and use in source and binary forms, with or without
% modification, are permitted provided that the following conditions
% are met: 1) Redistributions of source code must retain the above
% copyright notice, this list of conditions and the following
% disclaimer. 2) Redistributions in binary form must reproduce the above
% copyright notice, this list of conditions and the following
% disclaimer in the documentation and/or other materials provided
% with the distribution. 3) Neither the name of Yegor Bugayenko nor
% the names of other contributors may be used to endorse or promote
% products derived from this software without specific prior written
% permission.
%
% THIS SOFTWARE IS PROVIDED BY THE COPYRIGHT HOLDERS AND CONTRIBUTORS
% "AS IS" AND ANY EXPRESS OR IMPLIED WARRANTIES, INCLUDING, BUT
% NOT LIMITED TO, THE IMPLIED WARRANTIES OF MERCHANTABILITY AND
% FITNESS FOR A PARTICULAR PURPOSE ARE DISCLAIMED. IN NO EVENT SHALL
% THE COPYRIGHT HOLDER OR CONTRIBUTORS BE LIABLE FOR ANY DIRECT,
% INDIRECT, INCIDENTAL, SPECIAL, EXEMPLARY, OR CONSEQUENTIAL DAMAGES
% (INCLUDING, BUT NOT LIMITED TO, PROCUREMENT OF SUBSTITUTE GOODS OR
% SERVICES; LOSS OF USE, DATA, OR PROFITS; OR BUSINESS INTERRUPTION)
% HOWEVER CAUSED AND ON ANY THEORY OF LIABILITY, WHETHER IN CONTRACT,
% STRICT LIABILITY, OR TORT (INCLUDING NEGLIGENCE OR OTHERWISE)
% ARISING IN ANY WAY OUT OF THE USE OF THIS SOFTWARE, EVEN IF ADVISED
% OF THE POSSIBILITY OF SUCH DAMAGE.
%
\documentclass[12pt]{article}
\usepackage{amsmath}
\begin{document}
\title{Source Code Volatility (SCV)\\to Spot Dead Code}
\author{Yegor Bugayenko}
\maketitle

\section{Introduction}

Volatility of source code is an experimental metric that
shows how big is the difference between actively and rarely changed (possibly dead)
code. It is assumed that a big percentage of dead code is
an indicator of maintainability problems in the project.

\section{Details}

First, by looking at Git history,
it is observed how many times every source code file $i$ out of $N$ was touched
during the lifetime of the repository:

\begin{eqnarray}
T = [t_1, t_2, \dots, t_N]
\end{eqnarray}

Then, $t$ that relate to the files already absent in the
repository are deleted and the array $T$ is ``normalized''
to keep all values within the $[0,1]$ range:

\begin{eqnarray}
X = [x_1, x_2, \dots, x_M],\quad \text{where}\ x_i = \frac{t_i - \hat{T}}{\check{T} - \hat{T}}
\end{eqnarray}

Then, the mean $\mu$ is calculated as:

\begin{eqnarray}
\mu = \frac{1}{M}\sum{x_i}
\end{eqnarray}

Finally, the variance is calculated as:

\begin{eqnarray}
Var(x) = \frac{1}{M}\sum{|x_i - \mu|^2}
\end{eqnarray}

The variance $Var(x)$ is the volatility of the source code. The smaller
the volatility the more cohesive is the repository and the smaller
the amount of the dead code inside it.

\end{document}
